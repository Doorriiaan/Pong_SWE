Der letztendliche SOLL-Zustand ergibt sich final aus den \hyperref[sec:requirements]{Requirements} (Kapitel \ref{sec:requirements}).
Im Allgemeinen läst sich die Applikation wie folgt zusammenfassen:

Die App muss aus einem Startmenü, Spiel, Store, Highscore und Credits Screen bestehen.
Vom Startmnü Screen, müssen Spiel, Store, Highscore und Credits durch Buttons erreichbar sein. Auf jedem Screen muss es einen zurück Button geben.
Das Spiel muss Singleplayer und Hochkant sein.
Es muss aus zwei Spielmodi: "Classic" und "Invasion" ausgewählt werden können.
Classic und Invasion sollen aus den klassischen Elementen: Ball und Balken bestehen. 
Im Invasion Modus müssen die Kästchen eine bestimmte Anzahl von Malen getroffen werden, um zerstört zu werden.
Des Weiteren, muss es PowerUps geben.
Pro Spielmodus soll es 3 verschiedenen Schwierigkeitsstufen geben: "Easy", "Medium" und "Hard".
Unabhängig von Modus muss das Spiel jederzeit durch einen Button pausierbar sein.
Sobald der Ball unterhalb des Balkens ist, verliert der Spieler ein Leben. 
Wenn alle drei Leben aufgebraucht sind, muss es eine einmalige Chance geben, ein weiteres Leben durch das Anschauen einer Werbung zu bekommen.
Falls der Spieler einen neuen Highscore erreicht hat, muss es nach dem Spiel die Möglichkeit geben seinen Namen einzutragen.
Nach jedem Spiel muss dem Spieler sein Score und die verdienten Coins angezeigt werden.
Im Store muss man mit Coins, Skins für Ball, Ball-Schweif, Balken und Hintergründe kaufen und auswählen können.
Der Highscore Screen muss die Top 10 Scores, Namen, Spielmodi und Schwierigkeitsstufen anzeigen.
Im Credits Screen müssen die Namen von Personen, die an der Entwicklung der App gearbeitet haben und das Copyright, angezeigt werden.
