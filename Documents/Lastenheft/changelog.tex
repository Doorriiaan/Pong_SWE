Die folgenden Änderungen wurden im Laufe der Entwicklung des Lastenhefts vollzogen. Diese erfolgten aus den Gesprächen und dem Schriftverkehr zwischen Kunden (siehe \hyperref[sec:auftraggeber]{Auftraggeber}) und dem Vertrieb (Hristomir Dimov \& Leon Wiesen).

\begin{xltabular}{\textwidth}{|c|X|}
    \hline
    \textbf{Version}   & \textbf{Änderungen}     \\
    \hline

    v0.1    &  \begin{itemize}
        \item Kap. \ref{sec:auftraggeber}: Hinzugefügt
        \item Kap. \ref{sec:zeitbudget}: Hinzugefügt
        \item Kap. \ref{sec:bestimmung}: Hinzugefügt
        \item Kap. \ref{sec:einsatz}: Hinzugefügt
        \item Kap. \ref{sec:funktionen}: Neue MockUps
        \item Kap. \ref{sec:produktdaten}: Hinzugefügt
        \item Kap. \ref{sec:performance}: Hinzugefügt
        \item Kap. \ref{sec:quality}: Hinzugefügt
        \item Kap. \ref{sec:requirements}: Requirements erweitert
    \end{itemize}
    \\ \hline

    v1.0   & \begin{itemize}
        \item Allgemeine Rechtscheib- \& Grammatikfehler korregiert
        \item Glossar: Neue Begriffe Hinzugefügt
        \item Deckblatt: Gruppenname vervollständigt
        \item Kap. \ref{sec:auftraggeber}: Spezifischer formuliert
        \item Kap. \ref{sec:zeitbudget}: Glossar-Verlinkungen
        \item Kap. \ref{sec:einsatz}: Use-Case Diagramm hinzugefügt
        \item Kap. \ref{sec:funktionen}: Neue Diagramme
        \item Kap. \ref{sec:funktionen}: Genauere Beschreibungen
        \item Kap. \ref{sec:funktionen}: Glossar-Verlinkungen
        \item Kap. \ref{sec:performance}: Minimierung auf das Nötigste
        \item Kap. \ref{sec:abnahme}: Hinzugefügt
    \end{itemize}
    \\ \hline

\end{xltabular}